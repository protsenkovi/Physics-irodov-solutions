\testCom{3.11}
{Точка движется в плоскости xy по закону \(x = A\sin \omega t  ,  y = B \cos \omega t , \) где \( A,B,\omega\)  - постоянные.
Найти: \newline 
 \begin{inparaenum}[\itshape a\upshape)] 
	 \indent \item уравнение траектории точки \(y(x)\) и направление её движения по этой траектории;\newline
	 \indent \item ускорение \( a \) в зависимости от её радиуса-вектора \( r \) относительно начала координат.
\end{inparaenum}}
{%Дано
\(x = A\sin \omega t ,\)\\
 \(y = B \cos \omega t , \)\\
 \( A,B,\omega = const\)
}
{%Найти
\(y(x)\)-?\\
\( a(\bar{r}) \) - ?
}
{%Решение
a)\\
\( x = A \sin \omega t \)\\
\( \sin^2 \omega t = \frac{x^2}{A^2}\)\\
\( \cos^2\omega t = 1 - \frac{x^2}{A^2}\)\\
\(I = B \cos \omega t\)\\
\( \cos^2 \omega t = \frac{y^2}{B^2}\)\\
\(1 -  \frac{x^2}{A^2} = \frac{y^2}{B^2}\)\\
\(\frac{x^2}{A^2} +\frac{y^2}{B^2} =1 \)\\

$t=0 \Rightarrow$\\
$x=A\sin \omega * 0 = 0$\\
$x=B\cos \omega * 0 = B$\\

b)\\
\(\vec{r}=A\sin \omega t * \vec{i} + B \cos \omega t * \vec{j} \)\\
\(\dot r=A \omega \cos \omega t * \vec{i} - B \omega \sin \omega t * \vec{j} \)\\
\(\ddot r= - A \omega^2 \sin \omega t * \vec{i} - B \omega^2 \cos \omega t * \vec{j} \)\\
\(\ddot r= -\omega^2(A \sin \omega t * \vec{i} + B \cos \omega t * \vec{j} \)\\
\(\ddot r = -\omega^2 * \vec{r} \)\\
}

